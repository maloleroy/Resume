\sffamily

\begin{center}
    % NAME
    {\Huge\bfseries\sffamily \scalebox{1.0}[1.2]{Malo Leroy}}
    % SUBHEADING
    \\ Cherche un stage en Informatique Quantique -- Disponible Août 2025 - Février 2025 \\
    \small
    \href{mailto:malo.leroy@student-cs.fr}{\raisebox{-0.2\height}\faEnvelope\  malo.leroy@student-cs.fr} ~
    \href{tel:+33767553514}{\raisebox{-0.2\height}\faPhone\  {+33 7 67 55 35 14}} ~
    \href{https://linkedin.com/in/leroy-malo}{\raisebox{-0.2\height}\faLinkedin\ linkedin.com/in/leroy-malo}  ~
    \href{https://github.com/\ghusername}{\raisebox{-0.2\height}\faGithub\ github.com/\ghusername}
\end{center}


\section{Experience}
\resumeSubHeadingListStart

    \resumeSubheading
    {Software Engineer}{February 2025 - July 2025}
    {Highly selective gap year program}{Rust, C++, Docker\ghicon{fhe-bpce}}
    {Paris Digital Lab}
        \resumeItemListStart
            \resumeItem{Developed 3 MVPs, each within a 7-week constraint, to assist companies with their IT needs}
            \resumeItem{Programmed a peformance-focused, Cloud-native Fully Homomorphic Encryption framework for BPCE Group}
        \resumeItemListEnd

    \resumeSubheading
    {Quantum Computing Research Intern}{November 2023 - February 2025}
        {Part-time intern in INRIA's QuaCS team}{QASM, C++, GoogleTest, CMake\ghicon{coto}}
        {INRIA}
        \resumeItemListStart
            \resumeItem{Theorized an innovative data structure to represent quantum states \& circuits}
            \resumeItem{Started writing a scientific paper on the subject, detailing the mathematical background and properties of the data structure}
            \resumeItem{Created and implemented algorithms for fast classical simulation of quantum circuits}
        \resumeItemListEnd

    \resumeSubheading
    {Core Member}{September 2023 - February 2025}
        {Technical administrator of France's largest student ISP}{Kubernetes, Ansible, OpenStack, Ceph}
        {ViaRézo (IT association)}
        \resumeItemListStart
        \resumeItem{Operated 400+ Unifi APs and 50+ Juniper switches to provide Internet to 2.5k+ students}
        \resumeItem{Administered a 200+ virtual machines OpenStack cluster and a 150+ pods Kubernetes cluster}
        \resumeItemListEnd

        \resumeSubheadingContinue
        {Head of training}{Git, Python, \LaTeX, Docker, Ansible}
        \resumeItemListStart
            \resumeItem{Devised and conducted public training sessions on Git, advanced Python, and \LaTeX}
            \resumeItem{Developed in-house training programs on Bash, Ansible, and Docker to upskill team members}
        \resumeItemListEnd

        \resumeSubheadingContinue
        {General Secretary}{GDPR, NIS2, PostgreSQL, MySQL}
        \resumeItemListStart
            \resumeItem{Managed administrative and legal operations, ensuring compliance with national \& european regulations}
            \resumeItem{Held responsibility for database administration across all services (13M+ rows)}
        \resumeItemListEnd

    \resumeSubheading
    {Network Infrastructure Intern}{June 2024 - July 2024}
        {\resumeItemBackwards{Troubleshooted and secured a university library's network infrastructure}}{Cisco IOS, Wireshark}
        {Paris-Saclay University}
        \resumeItemListStart
            \resumeItem{Configured and deployed Cisco switches and APs in a multi-VLAN environment}
        \resumeItemListEnd

\resumeSubHeadingListEnd


\section{Projets}
\resumeSubHeadingListStart
    \resumeProjectHeading
    {Métaheuristiques pour le problème TSPTW}{Janvier 2024}
    {Developpement \& et benchmark exhaustif d'une solution approximative pour le problème TSPTW}{Jupyter, Numba}
    {}
    {CentraleSupélec}

    \resumeProjectHeading
    {Informatique distribuée pour la robotique}{Juin 2024}
    {Programmation d'un robot dont les parties opéraient indépendemment et communiquaient en réseau}{Python, Rust, Zenoh}
    {}
    {Zettascale}

    \resumeProjectHeading
    {Bibliothèque de tests unitaires}{\quad Août 2022}
    {Développement d'une bibliothèque rapide et sécurisée utilisant habilement le préprocesseur}{C, GitHub Actions\ghicon{confer}}
    {}
    {Projet personnel}
    \vspace{-7pt}
    \resumeItemListStart
    \resumeItem{Réalisation de pipelines pour l'intégration continue et le déploiement en direct de la documentation}
    \resumeItemListEnd

    \resumeProjectHeading
    {Simulation classique d'algorithmes quantiques}{Février 2022 - Juin 2022}
    {Création d'une bibliothèque formelle d'algèbre pour réaliser des calculs scalaires et matriciels exacts}{Python\ghicon{TIPE}}
    {}
    {Lycée Chateaubriand}
    \vspace{-7pt}
    \resumeItemListStart
    \resumeItem{Programmation d'une implémentation autonome de l'algorithme de Shor sur un maximum de 10 qubits}
    \resumeItemListEnd

    \resumeProjectHeading
    {Lecteur MP3 et lecteur de méta-données}{Juin 2021}
    {Implémentation d'une bibliothèque de lecture de méta-données MP3 directement à partir du standard ID3}{Flutter, Dart\ghicon{eanic}}
    {}
    {Projet personnel}
    \vspace{-7pt}
    \resumeItemListStart
    \resumeItem{Création d'un lecteur MP3 avec une interface utilisateur fonctionnant sur Android, iOS, Linux et le web}
    \resumeItemListEnd

\resumeSubHeadingListEnd


% \section{Technical Skills}

\vspace{-7pt}
\begin{itemize}
[leftmargin=0.15in, label={}]\small{\item{
    % LANGUAGES
    \textbf{Languages}{: French (native), English (fluent), Espagnol (conversational)} \\
    % LANGUAGES
    \textbf{Computer science languages}{: C++ (2 years), Python (6 years), C (4 years), Rust, JavaScript, Dart, SQL, \LaTeX} \\
    % FRAMEWOKS
    \textbf{Frameworks \& tooling}{: CMake, GoogleTest, Doxygen, Cargo, React, ExpressJS, Flutter, Django, Numpy} \\}}
\end{itemize}


\section{Formation}
\resumeSubHeadingListStart

\resumeSubheadingNoLocation
{CentraleSupélec -- Université Paris-Saclay}{2023-2027}
{Cursus ingénieur généraliste, GPA 3,97}{}
    \resumeItemListStart
    \resumeItem{CentraleSupelec est l'une des meilleures écoles d'ingénieur françaises, intégrée à l'Université Paris-Saclay (top 12 au classement de Shanghai des meilleures universités mondiales)}
    \resumeItem{\textbf{Cours pertinents:} réseaux \& sécurité, théorie de l'information, microarchitectures \& assembleur, algèbre, conception et vérification de systèmes critiques, physique quantique, informatique quantique}

    \resumeItemListEnd

\resumeSubheadingNoLocation
{Lycée Chateaubriand}{2021-2023}
{Class préparatoire aux grandes écoles}{}
    \resumeItemListStart
    \resumeItem{Classe préparatoire MPSI puis MP*}
    \resumeItemListEnd

\resumeSubHeadingListEnd


\vfill

\faGlobe ~ \textbf{Languages}{: French (native), English (fluent), Spanish (conversational)} \\
